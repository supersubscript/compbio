%TEX root = ../thesis.tex
%*******************************************************************************
%****************************** Second Chapter *********************************
%*******************************************************************************

\chapter{Methodology}

\ifpdf
\graphicspath{{Chapter2/Figs/Raster/}{Chapter2/Figs/PDF/}{Chapter2/Figs/}}
\else
\graphicspath{{Chapter2/Figs/Vector/}{Chapter2/Figs/}}
\fi

\section[Raw data]{Data consists of \textit{in vivo} confocal timelapses}
Six plants, labelled \textit{plant 1},
\textit{2},\textit{4},\textit{13},\textit{15}, and \textit{18}, were grown on a
solution consisting of 10$\mu M$ auxin transport inhibitor NPA to a depth of
roughly 1 cm for 22-26 days. The inhibition of auxin prevents formation of new
primordia, and this gives rise to a small and naked, organ-free meristem which is
tractable for imaging. 

The plantlets were marked with pUBQ10::acyl-YFP, which localises in the cell
membrane \CITE, as well as with pCLV3::dsRED-N7, which was used as a nuclear tracker
for CLV3 mRNA expression. Also pPin1::PIN1-GFP was tracked, but not quantified
in this study. In addition, \textit{plant 1} did not express the nuclear marker
for CLV3.

Using confocal microscopy, the six plantlings were tracked in
intervals of 4 hours up to 76 (\textit{plants 1, 2, 4}) or 84 hours
(\textit{plants 13, 15, 18}), using a 63x/1.0 N.A. water immersion objective.
Due to the high resolution of the images, the acquisition of each z-stack took
$\sim$10 minutes, which induced vertical stretching in the images due to stem
elongation. Because of this, a second batch of z-stacks was acquired, using
low-resolution imaging over $\sim$10 seconds. The original images were then
corrected, using this second batch as reference. \CITE 

% Construction of a YFP Plasma Membrane Marker and Other Transgenic Lines.
% DNA containing the coding sequence for YFP was amplified by PCR using
% primers attb1-mYfwd (5′-AAAAAGCAGGCTATGGGAGGATGCTTCTCTAAGAAGGTGAGC)
% and attb2-YFPrev (5′-AGAAAGCTGGGTTTACTTGTACAGCTCGTCCATGCCGAGAGTG).
% The total reaction volume was 50 μL. The forward primer contains a short
% sequence encoding a motif that is acylated in plant cells (54). Both primers
% contain a portion of the attB gateway sites. Amplification conditions were
% 96 °C for 1 min followed by 25 cycles of 96 °C for 30 s, 54 °C for 55 s, and
% 72 °C for 30 s, and a final elongation of 72 °C for 30 s. After checking for
% products on a gel, 5 μL of the PCR was used in a second reaction (40 μL total)
% containing primers B1 adapt (5′-GGGGACAAGTTTGTACAAAAAAGCAGGCT)
% and B2 adapt (5′-GGGGACCACTTTGTACAAGAAAGCTGGGT).
% Amplification conditions were 95 °C for 2 min followed by five cycles of
% 94 °C for 30 s, 48 °C for 30 s, and 72 °C for 1 min; 20 cycles of 94 °C for 30 s,
% 55 °C for 30 s, and 72 °C for 1 min; and a final elongation of 72 °C for 1 min.
% Products were PCR-purified (Qiagen) and then used in a one-tube format
% Gateway reaction as per the manufacturer’s instructions, with the destination
% vector pUB-DEST containing the UBQ10 promoter upstream of the
% Gateway site (55). The resulting vector, pUBQ10::acyl-YFP, was transformed
% into A. thaliana Col-0 containing pPIN1::PIN1-GFP (56, 57). The pUBQ10::acylYFP/pPIN1::PIN1-GFP
% plants were taken to the second filial (F2) generation
% and crossed with pCLV3::dsRED-N7 (58), a nuclear-localizing reporter for
% CLAVATA3 expression. This cross was taken to the F3 generation, yielding
% pUBQ10::acyl-YFP/pPIN1::PIN1-GFP/pCLV3::dsRED-N7 A. thaliana seeds. The
% pUBQ10::acyl-YFP reporter localized strongly and uniformly to cell membranes;
% it was stably expressed without cellular internalization and without
% affecting plant growth or development. These features permitted the accurate
% segmentation and tracking of cells. The reporter pCLV3::dsRED-N7, a
% nuclear-localized CLAVATA3 reporter, identified the stem cell niche’s center
% and, in a subset of SAM cells, enabled nuclear volume quantification (59) (SI
% Appendix, Supplemental Materials and Methods). The pPIN1::PIN1-GFP reporter
% was not analyzed as part of this study.
% Plant Growth Conditions. NPA-treated pUBQ10::acyl-YFP/pPIN1::PIN1-GFP/
% pCLV3::dsRED-N7 A. thaliana Col-0 plants were grown on plates with Arabidopsis
% medium supplemented with 10 μM NPA (31) at 20 °C with 16 h of
% light per day. These plants were later selected for imaging between 24 and
% 28 d after germination. NPA was used to inhibit organ formation (31)
% without substantially slowing proliferation in the SAM’s central zone (15) so
% that time-lapse images could be acquired without dissection, and therefore
% with minimal disturbance to cell proliferation.
% Time-Lapse Image Acquisition and Quantification. NPA-grown plantlets with
% naked, organ-free meristems were selected and gently transferred to lidded
% boxes measuring 5 × 5 × 3 cm3 containing room-temperature Arabidopsis
% medium supplemented with 10 μM NPA to a depth of ∼1 cm. Plantlets were
% screened for the expression of pUBQ10::acyl-YFP, pPIN1::PIN1-GFP, and
% pCLV3::dsRED-N7 using confocal microscopy, and then left to recover for
% 12 h in the same 16/8-h light/dark cycle. All three reporters were expressed in
% each of SAMs 2–6; SAM 1 expressed only pUBQ10::acyl-YFP and pPIN::PIN1-
% GFP. Confocal z-stacks were acquired every 4 h for 3–3.5 d at a resolution of
% 0.22 × 0.22 × 0.26 μm3 per voxel using a 63×/1.0 N.A. water immersion objective;
% excitation wavelengths of 488 nm and 561 nm; the corresponding
% dichroic filters; and a precalibrated spectral unmixing that enabled accurate
% separation of the YFP, GFP, and RFP signals. The confocal scan speed was no
% more than 9, and line averaging was set to 2. Each z-stack took ∼10 min to
% acquire. At the end of each high-z-resolution z-stack acquisition, a second
% low-z-resolution z-stack was rapidly acquired over ∼10 s with a z-step of
% 5–6 μm (to enable correction of a major artifact, a stretching in the z-direction
% owing to growth/movement in the stem during image acquisition; SI Appendix,
% Supplemental Materials and Methods). Data on cell size and growth kinetics
% were extracted by application of our 4D cellular quantification and
% tracking pipeline using MARS/ALT software (60) (SI Appendix, Supplemental
% Materials and Methods and Movies S5–S7).
% Statistical Analysis, Modeling, and Simulations. Cellular quantification and
% tracking data were analyzed with Python 2.7 scripts using the NumPy and
% SciPy libraries and StatsModels package. Simulations were performed based
% on a generalization of the models originally proposed (4, 37); simulations are
% detailed in SI Appendix, Text S3.

\section[Processed Image data]{Image Pre-processing and Segmentation}
\todo{Add figure of segmented plants}
In order to eliminate segmentation errors, the ImageJ plugin StackReg was used
to perform a translation transformation for each stack. Individual slices which
contained horizontal shifts because of vibrations or other types of system
disturbances were identified and replaced with the nearest slice that contained
no such shift. The z-directional stretching due to stem elongation was corrected for by mapping
the low-resolution stacks to the high-resolution ones in order to attain
stretching factors that the images were thereafter corrected for. 

For the membrane channel, noise removal was done by Gaussian and alternative-sequential
filtering. The filtered z-stacks were then watershed in 3D using the
algorithm implemented in the segmentation software \MA. Segmentation and
tracking was thereafter done using the same software. Cellular volumes were from
this then calculated as the sum of voxel volumes belonging to the same cell.
The tracking, also performed using \MA, was assessed for quality using an F1
score between the parent and corresponding daughter cell. For all analyses
discussed in this report, a cutoff value of 0.30 was set for the tracking in
order to account for likely incorrect mappings. These cells are included in the
overall analysis, but excluded from all cell line related investigations.

A longer outline of this is presented
in \cref{sec:data_errors}. 

The nuclear data were deconvolved to account for the microscope's point-spread
function using the \textit{PSF distiller} tool from Huygens software 15.05
\CITE. As in the membrane case, the nuclear channels were adjusted with the
corresponding stretching factors and thereafter segmented using segmentation
tool \textit{Costanza} \CITE. 

% Nuclear and membrane mapping
In order to link nuclei to membranes, the coordinates of each variable were
fitted using a least square approach. Duplicately mapping nuclei were then
consolidated as described in \cref{sec:filtering}. Note that whenever spatial
coordinates are referenced in relation to membrane or nuclei, they are done so
as the centroid coordinates of the basin of attraction found in the segmentation
process.

% Distance measures (d2t / circ)
Measures of distances to the top were done in multiple ways. The three methods
herein considered consist of a definition of the top based on 1) the spatial
coordinates, 2) the expression value, and 3) a least-square fit of a parabloid to
raw meristem images. In the case of spatial coordinates, the average $x-y$
coordinates of $n$ nuclei were chosen, complemented with the highest $z$ value
registered in the corresponding timeframe. For the second case, the apex was
defined as the $n$ average spatial coordinates of the highest expressing CLV3
nuclei. Lastly, the parabloid fit to the meristem was used to define the apex by
taking the coordinates of the region have a zero-valued derivative. For both the
segmentation-dependent approaches, the data was set to exclude subepidermal
layers in order to prevent biases.
\todo{Correct for overshooting?}

In order to achieve a cell-resolution description of distances in the SAM,
an auxilliary measure of cell distances was used in the form of a cell-wise
grouping. In the cell value utilising definitions of the apex above, cells
included in the definition was set to have a cell-wise distance of 0. The
neighbours of these cells were in turn defined to have a distance value of 1,
and so on recursively. 

\section[Data Filtering]{Data Filtering}
\label{sec:filtering}
Due to thresholding effects for cell nuclei during the segmentation, some
individual nuclei are
occasionally identified as two or more. In order to accound for this, nuclei
were mapped to the corresponding membranes using a minimum euclidian distance
measure between the respective centroids. The nuclear quantified metrics were
then corrected using the functions found in \cref{tab:consolidation_methods}. In
addition to this, all mentions of numbers of nuclei are with respect to the
number of cell membranes containing at least one nuclear volume identified
within them.

\begin{table}
  \centering
  \caption{Consolidation methods applied for duplicate nuclei.}
  \label{tab:consolidation_methods}
  \begin{tabular}{lc}                                 \\ \toprule
    \textbf{Metric}       & \textbf{Summary function} \\ \midrule
    Coordinates (x, y, z) & mean                      \\
    Nuclear volume        & sum                       \\ 
    Nuclear expression    & mean                      \\ \bottomrule
  \end{tabular}
\end{table}

For the data analysis section, data was excluded due to apparent segmentation
errors. This was done for each plant in isolation, with the outline of the
filtering described below in \cref{tab:filtering}. The choice of allowed
deviance was done based on the distribution shape \unsure{figure of these?}, with particular consideration
taken to the nuclear and membrane volumes, where no lower boundary was set. The
maximal neighbour distance was chosen due to the typical lack of data for cells more
than $7$ cell distances from the apex. Lastly, due to division events where
loss of nuclear signal took place, we filter out expression values which are
less than 70~\% of the magnitude in the previous, as well as subsequent timepoint.

\begin{table}
  \centering
  \begin{tabular}{ll}    \toprule
    Parameter & Value \\ \midrule
    Maximal membrane volume & $\mu + 3\sigma$ \\
    Minimal membrane volume & 0 \\
    Maximal nuclear volume  & $\mu + 5\sigma$ \\
    Minimal nuclear volume  & 0 \\
    Maximal apical distance & $\mu + 3\sigma$ \\
    Maximal neighbour distance & $7$ \\
  \end{tabular}
  \caption{Filtering settings}
  \label{tab:filtering}
\end{table}

For reference, a quality assessment of the data can be found in
\cref{sec:data_errors}. In addition, descriptions of the software used in
segmentation and tracking are briefly introduced in \cref{sec:software_descr}.

\section[Models of Gene Regulatory Networks]{Models of Gene Regulatory Networks}
\subsection[Mathematical Formulation]{Mathematical Formulation of Biochemical
  Reactions}

\subsubsection[Mass-action Kinetics]{Mass-action Kinetics}
The formulation of processes in GRNs focus primarily on two aspects: synthesis
and degradation of matter, which usually takes the form of molecular
concentrations or absolute abundance. As implied in \cref{sec:modelling}
(modelling biological systems), we here work using an ODE or SDE description of
our regulatory systems. 

We in this thesis represent our molecular reactions using mass-action
interactions and Michaelis-Menten kinetics; both here relying on the na\"ive
assumption that our primary reagents here act in isolation of other possibly
intervening molecules. As part of our formalism, we write
\begin{equation}
  S \xrightleftharpoons[k_{f}]{k_{b}} P
  \label{eq:simple_reac}
\end{equation}
to express that some substrate $S$ is turned into a product $P$ by some given
\textit{forward affinity} $k_f$. Likewise, as the reaction is \textit{reversible}, the
product $P$ is transformed back into $S$ with the \textit{backward affinity}
$k_b$. 

The \textit{law of mass action} states that the rate of a reaction
is proportional to its affinity, e.g.\ $k_f$, and the concentration of the
reacting species, here $S$. The reaction rate of the production of $P$ would thus be
$r_f = k_f S$. However, the rate of change of reactant $P$ also depends
on the backward affinity, which would give the overall rate-of-change for $P$ as
\begin{equation}
  \Delta P = k_f S - k_b P.
  \label{eq:mass_action_noninf}
\end{equation}
In the infinitesimal limit, we analogously have 
\begin{equation}
  \frac{\dd P}{\dd t} = k_f S - k_b P,
  \label{eq:mass_action_inf}
\end{equation}
i.e.\ on the form of a differential equation, which will be the baseline for our
formulations. Similar to the formulation of rate-of-change of $P$, we can do the
same for species $S$, and our system is then fully represented as a system of
differential equations.

Expanding on this, we can easily solve for the steady-state concentrations of
the system by assming that all rates average to zero. In our example above, this
gives us 
\begin{equation}
  \frac{k_f}{k_b} = \frac{P}{S},
  \label{eq:mass_action_ss}
\end{equation}
which holds in general, regardless of the number of reacting species.

\subsubsection[Michaelis-Menten Kinetics]{Michaelis-Menten Kinetics}
A conceptual drawback of the mass-kinetics formulation is the possibility to
have infinite reaction rates, whereas the molecular reactions in nature typically
are restricted by some means. One way to account for this is through
\textit{Michaelis-Menten kinetics}, which describes enzymatic chemical
reactions. In these, the trivial example introduced in \cref{eq:simple_reac} is
expanded to include an enzymatic agent such that
\begin{equation}
  E + S \xrightleftharpoons[k_f]{k_b} ES \xrightarrow{k_p} E + P.
  \label{eq:enzymatic_react}
\end{equation}
In other words, an enzyme-like molecule binds to the substrate $S$ such that the
complex $ES$ is formed. This complex is thereafter transformed into the product
molecule $P$ and again the enzyme $E$. Assuming a total enzyme concentration of
$E_{tot}$ and the assumption that the enzyme-substrate binding process is in
equilibrium, the rate-of-change of the product can be rephrased to be on the
form 
\begin{equation}
  \frac{\dd P}{\dd t} = V_{max} \frac{S}{K + S},
  \label{eq:michaelis-menten}
\end{equation}
where $K = K_f / K_b$ and $V_{max} = k_pE_{tot}$. This expression is said to be
on \textit{Michaelis-Menten} form, where $V_{max}$ is the maximal
activation rate of the protein, and $K$ can be thought of as a saturation
coefficient.

Extrapolating on this type of reaction, introducing $n$ enzymatically acting
molecules instad gives the standard \textit{Hill equation} form, namely
\begin{align}
  \frac{\dd P}{\dd t} &= V_{max} \frac{S^n}{K^n + S^n},\hspace{.8em} \text{and} \\
  \frac{\dd P}{\dd t} &= V_{max} \frac{K^n}{K^n + S^n}
  \label{eq:hill}
\end{align}
for an activating and repressing reaction respectively.

\subsection[Stochastic Simulations]{Numerically solving stochastic systems}
\subsubsection{Gillespie Algorithm}
The Gillespie algorithm is a discrete approach for simulating stochastic
molecular dynamics. It first appeared in print by Dan Gillespie in 1977, and has
since been widely used for stochastic simulations in multiple fields.

While being computationally expensive, the Gillespie algorithm compensates for
its lack in tractability by producing a statistically exact trace of the
molecular dynamics of a system. 

The algorithm originates in the formulation of the \textit{chemical master equation},
which specifies the rate of change of the transition probabilty between states
in the form of 
\begin{equation}
  \dfrac{\partial P\left( x,t | x_0, t_0 \right)}{\partial t} = 
  \sum_{j =
    1}^{M} \left[ a_j\left( x-v_j \right)P\left( x-v_j, t|x_0,t_0 \right) -
    a_j(x)P\left( x,t|x_0,t_0 \right) \right]
\end{equation}
where $a$ defines to reaction probability, or propensity, for each type of reaction, and
$v$ the stoichiometry, i.e. information of how the molecular species are changed
due to the reaction. $P(x,t | x_0, t_0)$ on its own denotes the probability of
$X(t) = x$, given that the initial value is $x_0$. Solving the master equation
analytically is usually complicated, so  
simulating a complex biological system using the Gillespie approach can often be
far more tractable.

Proceduraly, the algorithm can be formulated in four steps: 
\begin{description}
  \item[Initialisation] Generation of number of molecules and reaction
    parameters. 
  \item[Randomisation] Generation of random numbers to determine 1) next
    interaction, and 2) the time increment.
  \item[Update of system] Time and molecular numbers are update correspondingly
    to the determined event in step 2.
  \item[Repetition] Step 2-4 are repeated until some stop condition is met.
\end{description}

In principle, the Gillespie algorithm is interested in two fundamental
questions: 1) When does the next reaction happen? 2) Which is the next reaction?
The time until the next reaction at time $t$ is denoted $\tau$ and can be shown
to be an exponential distribution centered  
at $1 / \sum_{j=1}^{M}a_j(x)$ for some molecular concentration $x$, i.e.
\begin{equation}
  p(\tau = t') = \sum_{j=1}^M a_j(x)e^{-\sum_{j=1}^M a_j(x)t'}
  \label{eq:gill:time}
\end{equation}
with the reaction probability instead being described by the normalised
propensity. Historically, due to the limitation of random number generators, the
time update has been described as being drawn from 
\begin{equation} 
  \displaystyle
  \tau = \dfrac{1}{\sum_{j=1}^M a_j(x)} \ln\dfrac{1}{r_1}
  \label{eq:gill_time_update}
\end{equation}
although modern high-level programming languages do in some cases perform better using the
inherent random number generator for a specific type of distribution. \CITE

\subsubsection{Milstein's Method}
If there is no requirement for exactness, less computationally intense
alternatives to Gillespie's algorithm exists. One such example is the
\textit{Langevin} formulation of chemical systems, which utilises SDEs to
attain an approximate solution to the system trajectory, and is particularly
useful when the number of molecular reagents is high. 

The Langevin formulation, like Gillespie's, utilises the chemical master
equation to compute the behaviour of the system. In principle, the Langevin
formulation can be said to reformulate a deterministic increment of the form
\begin{equation}
  X_i(t + \dd{t}) = X_{i}(t) + \sum_{j=1}^M
  v_{ji}a_j(X(t))\dd{t}
  \label{eq:deterministic}
\end{equation}
to the stochastic form
\begin{equation}
  X_i(t + \dd{t}) = X_{i}(t) + \sum_{j=1}^M
  v_{ji}a_j(X(t))\dd{t} + \sum_{j=1}^M
  v_{ji}a_j^{1/2}N_j(t)\dd{t}^{1/2}
  \label{eq:stoch}
\end{equation}
where $X$ denotes the molecular number, $v$ the stochiometric coefficient of the
equation in question, and $N_j$ are temporally uncorrelated and statistically
independent, Gaussian random numbers with mean 0. From this stage, the equation
is then easily extended to its multivariate form, namely
\begin{equation}
  X_i(t + \dd{t}) = \sum_{j=1}^Mv_{ji}a_j(\bar x) \dd{t} +
  \sum_{j=1}^M v_{ji}{a_j}^{1/2}(\bar x) N_j(t) \left( \dd t \right)^{1/2}
  \label{eq:stoch_multi}
\end{equation}

Milstein's approach to solving this equation numerically utilises
\cref{eq:stoch} on the differential form 
\begin{equation}
  \dd X_t = a(X_t) + b(X_t) \dd W_t
  \label{eq:milstein_form}
\end{equation}
where $W_t$ is a continuous-time stochastic process. The simulation interval
$\left[ t_0, T \right]$is then partitioned into parts of size $\Delta t = T /
N$, where $N$ is the number of partitions. We thereafter define the update
\begin{align}
  Y_{n+1} &= Y_n + 
  a(Y_n)\Delta t + 
  b(Y_n)\Delta W_n +
  \dfrac{1}{2}b(Y_n)b'(Y_n)\left( \left( \Delta W_n \right)^2 - \Delta t \right) \\
  Y_{n+1} &= Y_n + 
  a(Y_n)\Delta t + 
  b(Y_n)\Delta W_n + 
  \dfrac{1}{2} b(Y_n)b'(Y_n) \left(\Delta W_n\right)^2 
  \label{eq:milstein_deriv}
\end{align}
on It\={o} and Stratonovich form respectively. Here $b'$ denotes the spatial
derivative of $b$, whereas $\Delta W_n = W_{\tau_{n+1}} - W_{\tau_n}$. The
difference between the It\={o} and Stratonovich form in turn is the
interpretation of the integral of $\dd W_t$. \CITE In many cases, it is
preferable to express the numerical update on a derivative-free form, which can
be done through a Runge-Kutta like approach \CITE and gives the final,
multivariate expression as 
\begin{equation}
  Y_{i, n+1} = Y_{i, n} + a_i\left( Y_{i,n} \right) \Delta t + 
  b_{ii}(Y_{i,n}) \sqrt{\Delta t} N_i + \dfrac{1}{2\sqrt{\Delta t}} \left[
    b_{ii}(\bar x, n) - b_{ii} \right]
  \Delta t (N_i)^2 
  \label{eq:milstein_deriv_free}
\end{equation}
where a supporting predictory step is calculated in the form of 
\begin{equation}
  \bar {x_i} = x_i + a_i\left( Y_n \right)\Delta t + b_{ii}\sqrt {\Delta t}.
  \label{eq:milstein_predictor}
\end{equation}
Algorithmically, the Milstein approach is of strong order of convergence
$\mathcal O \left( \sqrt{\Delta t }\right)$ and weak order $\mathcal O \left(
  \Delta t \right)$. In this thesis, we utilise the Milstein approach under the
Stratonovich interpretation.


\section{My models (come up with title)}
In this thesis, we model regulation of gene expression using hill equations for
production and exponential decay for degradation. Gene products are instead
set to undergo linear production with respect to the activating gene, as well as
diffusion. Like for gene expression, proteins are modelled using exponential
decay. In total, the equations governing the regulation for a gene $X$ and its
protein $x$ can then be formulated as

\begin{align}
  \frac{\dd X}{\dd t} = 
  \prod_{a=1} V_{max}\frac{X_a^{n_{a}}}{K_a^{n_a} + X_a^{n_a}}
  \prod_{r=1} \frac{K_r^{n_{r}}}{K_r^{n_r} + X_r^{n_r}} -
  dX, \ \ \ \text{and} \\
  \frac{\dd x}{\dd t} = pX + D\Delta x - dx,
  \label{eq:gene_expr_reg}
\end{align}
where $\nabla$ is the \textit{Laplace operator}.

% \section[Epidermal]{Epidermal Model}
% % Birth / death
% % Birth / death / self-activation
% 
% \section[Internal]{Internal Model}
% Will I even do this? 

