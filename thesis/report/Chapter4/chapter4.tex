%!TEX root = ../thesis.tex
%*******************************************************************************
%****************************** Third Chapter **********************************
%*******************************************************************************
\chapter{Discussion}

% **************************** Define Graphics Path **************************
\ifpdf
\graphicspath{{Chapter4/Figs/Raster/}{Chapter4/Figs/PDF/}{Chapter4/Figs/}}
\else
\graphicspath{{Chapter4/Figs/Vector/}{Chapter4/Figs/}}
\fi

\section{A periodicity is indicated in the fluctuations of CLV3 nuclei}
% NPA is diluted -- oscillations
It is likely that the longer term changes that appears in \cref{fig:clv3_trajs} are
caused by the slight dilution of NPA that occurs during imaging. Hypothetically,
when NPA is diluted, auxin transport should incrementally be restored in the plant
and cause initiation of primordia. In support of this, the fact that the
observed cyclicity appears to  
happen in a $\sim$16 hour interval suggests that the fluctuations are not
primarily driven by the circadian clock, which could otherwise be a cause of the
periodicity. Since it is well-known that the network regulating
the niche at the SAM consists of multiple negative feedback
loops~\cite{gordon2009multiple}, where
the signals have a natural delay by moving between cells, it is
possible that these are what causes the oscillatory tendencies as the
plants trends towards the new steady state induced by the NPA dilution.

% CLV3-WUS creating oscillations?
While some of the variations between plants can be attributed to biological
differences, the overall tendency of the four plants in their behaviour over
time indicate a biochemical response to the NPA dilution in a periodic manner. A
possible direct pathway for this type of effect could be through the direct WUS
or CLV1/CLV3 interaction of NPA, although more likely is downstream factors
being influenced by the initialisation of auxin patterns. Nevertheless, a
negative feedback motif responding to the system perturbations would be precisely the
type of mechanism which would be expected to drive a periodic response.

% Possible mechanism driving increased oscillations
That the fluctuations have an increasing amplitude as time progresses could be a
consequence of the iterative dilution of NPA. The claim that auxin transport is
being restored is emphasised by multiple plants beginning to form primordia throughout
the timecourse (\cref{fig:NPA_primordia}).

% Number of divisions increase
As the number of division events correlates clearly with the number of nuclei
identified, the perturbations in the system seem structural in the sense that
the stem cell niche grows not only in size, but also in what number of cells
actively proliferate. As the plant grows, we are therefore observing not only a
larger number of cells in general, but also fluctuations in number of cells that are
part of the effective stem cell niche.

\section{Distributional cues hint at possible regulatory mechanisms}
% WUS activating CLV3
The regulation of noise at the SAM is explained by our simple activating model,
where the up-regulation of CLV3 by WUS is sufficient to produce the
hysteresis-like features of the distribution. The reason for this is explained
by production saturation at the apex when CLV3 levels are already high, and the
highest activating concentration being present in this region, such that when
CLV3 experiences a minor stochastic degradation event, the activating signal
quickly overcomes this downgrading and resets the high expression in the apical
cells. In the model, the GRN creates a
region in the meristem where the transition between high and low levels of the
CLV3 activating agent causes high amounts of noise at intermediate distances from the apex. Because of
this, the enzymatic activation is sufficient to produce the CLV3 variability
pattern after just selecting parameters to fit the average distribution.
In the peripheral tissue, WUS is simply not abundant enough for producing a
high CLV3 concentration, which also results in a lower tail variance. 

While our simple model is sufficient to explain the type of distribution in the
epidermis, the apparent scaling in values due to deeper penetration in the
tissue suggests a mechanism for repressing CLV3 expression in this region, or
for activating the same in the epidermis. Although
both direct repression and repression of an activator such as WUS are
possible regulatory causes, previous research has suggested epidermal
signalling as a possible mechanism to scaling the stem cell
niche~\cite{gruel2016epidermis}. In addition, the
distribution differences between the epidermis and the subepidermal layers with
respect to the relationship between nuclear volume and CLV3 expression indicate
differing regulation between the layers, as we otherwise would expect to see the
same types of distributions, albeit scaled. Lastly, as the number of divisions seem to
quickly decline in cells beyond the fourth cell from the apex, it possible that
cells at that point are entering a regime which is under the influence of
differentiation  inducing signals, making them less prone to proliferate. In
support of this, the average membrane size per layer does not vary
significantly between epidermal cell layers more than one cell distance from the apex, 
while the division rate clearly declines for cells at distance $>3$. The data
thus suggests that molecular signalling, rather than directly size related ones, 
are what gives rise to the observed effect. Nevertheless, this could be
explained by different growth rates of the cells at these distances.

% Definition of apex
\section{\textit{In vivo} tracking sheds light on steady regulation of the true stem
  cell niche}
The behavioural discrepancy between the topmost cells and their
neighbours suggests a phenotypic separation between the roughly four highest
expressing cells and their neighbours. The fact
that some individuals disagree  might be due to the
absolute definition of which cells are the topmost ones, i.e.\ four in our
configuration. It is possible that e.g.\ plant 2 simply maintains a differently sized
stem cell niche, or is generally more variable in its expression levels, which
would introduce significantly more noise in the measure and explain the lack of
observed clusters.  

The \textit{in vivo} tracking of lineages in the L1 reveals insight into the
maintenance of CLV3 expression at the top. A select few cells are clearly
subject to highly regulated CLV3 expression values, whereas the cells outside of
this regime appear to be steadily degraded. The repeated signs of the apical
cells differing functionally however suggest that
a significant portion of the observed CLV3 expression stems from the
long half-life of dsRED, which is known to be able to last for extended periods
on the order of magnitude of several days, depending on biological
context~\cite{filonov2011bright}. Our analysis allows us to better  
approximate the boundaries of the CLV3 domain, as we can
estimate the potential degradation time of dsRED.  
While our estimate of the reporter half-life is highly approximate and hampered by
the lack of complete data, it shows promise for extended analyses of how dsRED
is degraded within the living meristem tissue. 

%Might be a good place to discuss errors and why we don't see stem cells in L2.

\subsection{Longevity could be induced by the WUS domain}
The observed cluster of cells not dividing for extended periods of time could
due to tracking errors and failure to identify division events,
which would cause an amassment of cells living for multiples of the average cell
cycle length. However, this type of effect would in principle cause a peak in
the division age density also at roughly the 40 hour mark, which we do not see
in \cref{fig:age}. The only clusters seen are instead emphasised around the
$~20$ and $~60$ hour mark, with a slight downwards shift for $L2$ and $L3$ due
to cells being lost in the tracking reappearing as new cells in the analysis. 
A possible factor affecting the results could be the penetration depth of the
laser in the deeper tissue, which due to the paraboloid-like shape of the
meristem could cause errors in tracking. Nevertheless, this would not explain
the lack of peaks at the $\sim$40 hour mark. It should also be noted that cell
walls might not be identified at certain timepoints, but show up at later ones,
and thus effectively broaden the upper tail of the division age distributions.
Still, we observe pronounced peaks for multiples of the average cell cycle in
the deeper tissue.  

The high longevity cells appear to be clustered to regions close to the
center of the meristem (\cref{fig:age_clusters}). A causation of this
could be an influencing factor in the OZ, e.g.\ WUS, which significantly
represses proliferation. This finding supports the previously reported
qualitative differences of cell division rates, and the high spatiotemporal
resolution in our data allows us to make a quantitative comparison.   

% Spatial cues from division data
\subsection{Non-centric CLV3 apex for directed development}
The textbook description is that that the CLV3 expression is at the very apex of
the meristem~\cite{clark2001cell}.  
That the CLV3 peaks typically tend to be shifted slightly from the geometric
apex could have a role in directing growth of the meristem. The fact that we
can note a seemingly normal distribution of distances from the apex for the CLV3
peaks could indicate a feedback between CLV3 peak localisation and growth, where
non-central CLV3 amassment could be used to specify the direction or focus of development.
A more speculative but highly interesting idea is
that this could influence the positioning of new organs in the meristem by a
continuous rotation around the geometric center. This type of effect would
potentially relate to the fact that AT organs are initiated sequentially in a
spiral-like structure.

It is also possible that we under correct angles would be able to see a
slight wiggling across the apex over time, although the lack of control of the
angle in which the plants were imaged makes this analysis difficult.  

%Paraboloid + definition of apex given n cells
