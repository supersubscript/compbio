%!TEX root = ../thesis.tex
%*******************************************************************************
%*********************************** First Chapter *****************************
%*******************************************************************************

\chapter{Introduction}  %Title of the First Chapter

\ifpdf
    \graphicspath{{Chapter1/Figs/Raster/}{Chapter1/Figs/PDF/}{Chapter1/Figs/}}
\else
    \graphicspath{{Chapter1/Figs/Vector/}{Chapter1/Figs/}}
\fi

%********************************** %First Section  **************************************
%\section{The Shoot Apical Meristem of \textit{Arabidopsis thaliana}} %Section - 1.1 

% first letter Z is for Acronyms 
% first letter A is for Roman symbols
% first letter G is for Greek Symbols
% first letter G is for Greek Symbols
% first letter G is for Greek Symbols
% first letter X is for Other Symbols
% first letter R is for superscripts
% first letter S is for subscripts
\nomenclature[z-CZ]{CZ}{Central zone}                                          
\nomenclature[z-SAM]{SAM}{Shoot Apical Meristem}                               
\nomenclature[z-RAM]{RAM}{Root Apical Meristem}                               
\nomenclature[z-CLV3]{CLV3}{CLAVATA-3}                                         
\nomenclature[z-WUS]{WUS}{WUSCHEL}                                             
\nomenclature[z-KAN]{KAN1}{KANADI-1}                                           
\nomenclature[z-AT]{AT}{\textit{Arabidopsis thaliana}}
\nomenclature[z-L1]{L1}{Layer-1. The outermost cell layer of the SAM}
\nomenclature[z-CZ]{CZ}{Central Zone. The region harboring stem cells in the SAM.}                                                                        
\nomenclature[z-GRN]{GRN}{Gene Regulatory Network}
\nomenclature[z-SLCU]{SLCU}{Sainsbury Laboratory at the University of Cambridge}

%********************************** %Second Section  *************************************
\section{The Shoot Apical Meristem of \textit{Arabidopsis thaliana}} %Section - 1.2
% The Shoot and the Root
Plant stem cells are governed by two developing centra -- the Shoot Apical
Meristem (SAM) and the Root Apical Meristem (RAM). The SAM is the region
responsible for development of all aerial organs in the plant, which includes
aspects of cell proliferation and specification, as well as an ability of the plant to
maintain and regulate the stem cell identity of the cells at the very apex of
the shoot. \CITE As opposed to the RAM, which has two stem cell pools in the
inside of the root, the SAM maintains a single stem cell pool centered at the
apex. It also lacks the root's cap, which protects the stem cells on the inside
of the root, whereas these in the shoot are directly exposed to the plant's surroundings. 

% What stem cells do
The stem cells at the SAM contribute to the construction of new organs and
general tissue by dividing frequently at the top and subsequently being
mechanically pushed out of the center in order to differentiate. The steady
maintenance of the stem cell niche allows for a constant production and supply
of cells that the plant utilises during both growth and repair of damaged
tissue. \CITE

% Structure composition of the plant
In a simple outline of the SAM, it can be said to consist of three core regions:
1. The \textit{central zone} (CZ), which harbors the aerial \textit{stem cell niche} 
of the plant; 2. The \textit{Rib Meristem}, which is located beneath
the CZ and consists of the cells constructing the stem of the plant; 3. The
\textit{Peripheral Zone} (PZ), where cells form organs and new tissue through
differentiation. In addition to these regions, the SAM is also often separated
into the different layers of the dermis, denoted \textit{L1} for the epidermal
layer, \textit{L2} for the sub-epidermal one, and \textit{L3} for the inner ground
and vascular tissues. For cells in both L1 and L2, proliferation happens
orthogonally to the shoot surface, i.e.\ so that cell lineages are preserved
within L1 or L2 correspondingly. In contrast, this is not the case for L3, where
cells can divide in all directions. In addition, it has been shown that the epidermis is
involved in both promoting and restricting shoot development, adding to the
notion of coordination and regulation between the different cell layers in order
to accurately direct plant growth. % Expand on this

\todo{Lead into modelling}

\section{Modelling Biological Systems} %Section - 1.2
% Systems biology
Due to the interaction of molecules in various ways, e.g. the activation or
repression of transcription by certain proteins, organismal development can be
considered in the framework of being a \textit{complex system}. In a
\textit{systems biology} setting, molecular and mechanical interactions are
treated as abstract entities, each representing some fundamental part of the
whole system in question, much like how machinery can be explained by its
separate cogs and gears working together. In a molecular setting, the typical
descriptive approach is through \textit{Gene Regulatory Networks} (GRNs), where
each component represents some molecular aspect of the system that is involved in
producing expression levels of mRNA, proteins and hormones. \CITE

% Gene regulatory networks and mathematical modelling
GRNs are commonly understood both through analytical and computational means,
where in the latter computer-generated models provides as a modern tool for
better understand the complex nature of many biological systems. Typically,
reaction kinetics are modelled using various types of \textit{Ordinary
  Differential Equations} (ODEs). However, due to the large supplies of computer
power available in the modern day, many recent studies also utilise more
computationally demanding resources such as \textit{Stochastic Differential
  Equations} (SDEs), where also the inherently random nature of molecular
motions, interactions, and processes are accounted for. In particular stochastic
modelling of biological systems able to capture dynamical features that
deterministic versions cannot. For example, cells often require in various ways to be
able to switch between active and inactive states, e.g. when commiting to
producing a certain protein or not. Utilising the inherent noise of microscopic
systems, cells have been shown to probabilistically tune their responses; \CITE 
modelling these types of phenomena using stochastic approaches has gained much
insight into the decision-making process of cells both specifically and in
general. \CITE 

In addition to stochasticity, the increase in computability has also allowed for the development of
spatiotemporal modelling,   where models are evaluated not only in a static
context, but also in a changing  
setting. A straightforward example taken directly out of the context of plant
development is how the distribution of gene expression varies during plant
growth and organ formation, both spatially and over time. Typical modelling
aspects at such a problem case involves the formulation of which genes and
molecules are imporant for the problem of interest, as well as how the
discretisation and representation of spatial elements is done. \CITE

% Model prediction and verification 
Computer models in general have two separate aims: exploration and verification.
In the former case, computer simulations can be the core for designing
experimental experiments, where observed theoretical phenomena can be
experimentally tested. An example of this is the classic example of the
\textit{Repressilator} \CITE, where researchers set up a model framework for how
oscillations could occur due to cyclic repressive interactions between three
genes. This was then verified to occur by synthetically implementing the system
in a bacterium, showing how gene expression profiles could oscillate due to the
system motif constructed. 

% TODO: 
In the latter case, which is the more prevalent in modern computational biology,
computer models are established in order to verify or support the potential of
an hypothesis due to experimental observations. \CITE 
\REWRITE

\section{Regulatory Mechanics of Plant Stem Cells} %Section - 1.2
\subsection{Molecular tuning determines cell phenotype} %Section - 1.2.1
% Cell decision making
The ultimate phenotype of a cell is to a large degree determined by the underlying
expressed genes and proteins, which in turn are regulated by the core GRN. Cells
that have not yet undergone the differentiation process are those which are
broadly described as stem cells. In addition to not having a specialised phenotype,
stem cells continuously proliferate in order to give rise to new cells that can
be used for development or repair. \CITE
Similar to in animals, stem cells require an intricate network both specifying
the pluripotency to the cell, and being able to maintain this both when the plant
conformation or the environment changes. Effectively, this regulation causes the
stem cell niches of the plant to be determined by various types of patterning,
which also plays a role in specifying zones of initiation of primordia.

% Patterning
A viable and robust network maintaining patterning is thus important for 
the plant in order undergo phyllotaxis in a functional manner, and to know when
and how to commit to more structural changes. Substances which rule this type of
process are known as \textit{morphogens}, and guide the initiation of organs and
specialised cells by signalling processes, where cells are tuned to respond
accordingly depending on its local configuration of molecular concentration.

Morphogen patterns can consist of several
types of spatiotemporal expression, including that of hormones, proteins and
RNA localisation, although in extension to molecular interactions, also patterns
of stress and strain have in recent studies been shown to play a role in
determining both growth and cell identity. 
\CITE (x2) Typically, whenever gene expression is the focus of a study, it
is often used as a proxy for protein expression, as fluorescent tagging
and tracking of proteins sometimes interfere with the function or transport of
the molecule. 

\subsection{Developmental Regulation in the Apical Meristem} %Section - 1.2.1
% WUS and CLV3
The GRN in the SAM is determined mainly by two core genes -- \textit{WUSCHEL} (WUS) and
\textit{CLAVATA} (CLV). Their corresponding network consists of the homeodomain
protein WUS and a ligand-receptor complex made up by CLV1 (receptor), CLV3 (ligand) and
an assumed accessory protein CLV2. In particular CLV3, which is expressed in a
few cells at the very apex CZ, correlates strongly with stem cell identity of
the cells. In these cells, CLV3 encodes a small, secreted peptide which diffuses quickly
out of the cell. The clv3 gradient extends down to the OZ where it acts
repressively on the WUS gene.\todo{Add information on CLV mutants}

\FIG \todo{Add flowchart of shoot network}

In contrast, WUS agonistically  
activates the CLV pathway through diffusion of its homeobox protein. This activating
interaction makes it necessary for maintaining an appropriate stem
cell niche, and for repressing the differentation process of the cells at the
apex. This is particularly noticable in WUS loss-of-function mutants, where the
lack of the correct WUS gradient leads to defective shoots that terminate in
aberrant flat structures. \CITE (Laux 1996) 

Together, the CLV3-WUS feedback loop forms the core of the GRN regulating stem
cell identity. Outside of the CZ, peripherally expressed genes such as KAN1 \CITE 
are known to promote cell differentiation. The core network itself is naturally
also affected by the activity of other genes in extension, including hormonal
intervenience on WUS by the small and diffusive hormone cytokinin, which itself
is activated by enzymes present in the meristem. Also other homeobox encoding
genes such as \textit{Shoot Meristemless} (STM) are essential for correct
development of the shoot. 

Aside of the system regulating the cellular
identity, even further substances are key to the overall development.
The plant hormone auxin in particular has been repeatedly shown to have an
essential role in the coordination of growth, both in 
signalling initiation points of new primordia and elongation of the core stem.
Because of this, auxin transport is key to asserting apical dominance in plants
through the help of active transporters such as the PIN-FORMED (PIN) family,
inhibition of this process leads to development of organless meristems.
\CITE 

As a whole, the molecular regulation of the
development of the SAM consists of an intricate system that requires both tight
regulation and precise coordination. This allows the plant to both counter and 
utilise noise that might be present due to volatile environements, or inherent
molecular processes, so that functional and robust development can be ensured.

\section{Background (etc.)}
An important question in developmenal biology is how organisms can have robust
development despite consisting of many independetly variable parts. \CITE (x3) \\
At the same time, genotypically similar plants can nevertheless exhibit
significant differences in phenotype, raising the questions of how, where and
why noise impacts the development of the plant. 

While plants, and AT in particular, have been studied throughly over the years,
it is not until recently where significant advances in imaging has allowed for
more fine-grained analyses both on 1) development of the plant \textit{in vivo},
and 2) the extent and regulation of noise during development. The modern
possibility of using 3D confocal microscopy to observe growth at the
single cell level has in this spirit opened the door for quantified analyses on
plant and cellular behaviour on the single cell level. Because of this, it is now
possible to expand on this using timelapses of confocal images taken under a
period to resolve not only the static image, but also the dynamic. 

Quantification of the SAM tissue at the single cell level\\
Stable regulation (and regulation \textit{per se})\\
CLV3 tracking and anisotropic growth \\



% A key feature of stem cells is their ability to maintain their population over time, effectively
% forming a replenishing source of building material for the plant throughout its lifetime. In the
% SAM, pluripotent cells are maintained in the central zone by a genetic feedback loop consisting
% of genes CLAVATA-3 (CLV3) and WUSCHEL (WUS), where the former is expressed in the
% central zone itself, and the latter in the uppermost parts of the rib meristem. The coordination
% between the different parts of regulation must be tightly organised in order to assure correct
% spatiotemporal formation of new tissues. Nevertheless, plants exhibit developmental variability
% in such a way that stochasticity appears to be conserved [2], which raises the unresolved ques-
% tion of how robust SAM development incorporates noisy expression. Using in vivo imaging data
% on multi-gene reporter systems, I hope to quantify the extent of noise in the regulatory network
% of the SAM, and subsequently develop computational models describing how this impacts cell
% division, differentiation and homeostasis throughout the life-span of the plant; all these vari-
% ables being factors which affect the final phenotype, and hence the ultimate functionality of the
% organism.

% Given that modern imaging tools now are capable of capturing high-resolution
% development both in the spatial and temporal sense, the need for
% models describing the underlying dynamics of these processes is rapidly
% increasing. The improvement in the attainable data also allows for far more  
% fine-grained analyses on, for example, the role of stochasticity in development,
% and in extension, how local homogeneity is maintained also in shifting  
% environments – that is, how cellular systems adapt to external perturbations. On
% the more macroscopic scale, the relevant question becomes how  
% fundamentally variable systems can give rise to stable heterogeneity in
% developmental processes. Specifically, I would like to research how stable  
% spatial gene expression patterning is maintained by stochastic processes, and
% how stochasticity can drive heterogeneous growth, which in the bigger  
% picture can have implications for more applied fields such as crop improvement
% and regenerative medicine.  
% 


% Maintaining stem cell homeostasis in the shoot and root
% stem cell niches is essential to ensure that an equal number
% of new cells are generated to replace those that are
% displaced from the niche, to differentiate and to enable
% the growth and formation of new tissues and organs.
% Remarkably, the RETINOBLASTOMA-RELATED
% (RBR) protein, the plant homologue of the RB tumour
% suppressor protein, has a crucial role in both niches3,11.
% As in animals, RBR inhibits cell cycle progression by
% interacting with an E2F transcription factor homologue12.
% Moreover, reduced levels of RBR result in an increase
% in stem cell numbers, and increased RBR levels lead to
% stem cell differentiation, which indicates a prominent
% role for RBR in stem cell maintenance13–15. At present,
% RBR is the only known protein involved in stem cell
% function that is conserved between the animal and plant
% kingdoms.
% Maintaining shoot stem cells. Maintenance of a stable
% shoot stem cell pool mainly involves a feedback loop
% between the homeodomain protein WUSCHEL (WUS)
% and a ligand–receptor signalling cascade, which is
% collectively known as the CLAVATA (CLV) pathway.
% Expression of the WUS gene defines the organizing
% centre and the WUS protein acts as a non-autonomous
% signal to maintain stem cells and, at least in meristems,
% is sufficient to promote stem cell identity16–21. Restricting
% WUS movement results in premature loss of stem cells,
% which indicates that its movement is important for stem
% cell maintenance22. WUS binds to and activates the promoter
% of CLV3, which encodes a signal peptide that is
% expressed in stem cells and, in turn, CLV3 signalling
% represses WUS transcription20,22–27 (FIG. 3a). However,
% although CLV3 is a marker for stem cells, clv3 mutants
% retain a stem cell population, which suggests that the
% CLV3 protein is not essential for stem cell specification, 


