%!TEX root = ../thesis.tex
% ******************************* Thesis Appendix A ****************************


\ifpdf
\graphicspath{{Appendix1/Figs/Raster/}{Appendix1/Figs/PDF/}{Appendix1/Figs/}}
\else
\graphicspath{{Appendix1/Figs/Vector/}{Appendix1/Figs/}}
\fi



\chapter{Data quality and errors} 

%\section{Experimental procedure} 
%The Yellow Fluorescent Protein (YFP) marker for the plasma membrane was
%amplified using PCR with primers attb1-mYfwd
%(5'-AGAAAGCTGGGTTTACTTGTACAGCTCGTCCATGCCGAGAGTG) and attb2-YFPrev
%(5'-AGAAAGCTGGGTTTACTTGTACAGCTCGTCCATGCCGAGAGTG), with the forward primer
%sequence containing a motif known to acetylate in plant cells~CITEHERE. 50
%$\mu$L solution was amplified in 96 $^\circ$C for 1 minute, followed by 25
%cycles of 96 $^\circ$C for 30 seconds, and a final elongation for 30 seconds. 5
%$\mu$L of the result was then used in a second reaction consisting of
%40~$\mu$L solution in total, with primers B1 adapt (5'-GGGGACAAGTTTGTACAAAAAAGCAGGCT)
%and B2 adapt (5'-GGGGACCACTTTGTACAAGAAAGCTGGGT) included. Similar to the first
%solution, the second mixture was amplified by PCR in 95 $^\circ$C for 2 minutes,
%followed by 94 $^\circ$C for 30 seconds, 48 $^\circ$C for 30 seconds, and 72
%$^\circ$C for 1 minute, 20 cycles of 94 $^\circ$C for 30 seconds,
%55 $^\circ$C for 30 seconds, and 72 $^\circ$C for 1 minute. Finally, elongation
%took place under 72 $^\circ$C for 1 minute.
%Membrane \\
%Nuclei \\
%Unnused data (PIN) \\


\section{Tracking errors}
\label{sec:data_errors}
\begin{figure}[p]
    \centering
        \centering
        \includegraphics[width=.95\textwidth]{tracking_quality_plant.pdf}
        \centering
        \includegraphics[width=.95\textwidth]{tracking_quality_time.pdf} % second figure itself
        \caption[Tracking quality]{Quality distributions per plant and timepoint for all plants.
        Some timepoints have distributions scoring in the $\sim$1 regime, which
        corresponds to situations where the automatic tracking failed. These
        timepoints have been manually tracked for cells in the L1 within 30
        $\mu$m of the topmost CLV3 expressing cell.}
      \label{fig:tracking_quality}
\end{figure}
The data tracking quality is defined as the F1 score, also known as the
Dice-S{\o}rensen score, between timepoints. The
F1 score gives an appreciation of the overall accuracy of a test on the form of
\begin{equation}
  F_1 = 2 \frac{precision\times recall}{precision + recall},
  \label{eq:f1}
\end{equation}
i.e.\ as the harmonic mean between precision and recall. It can also be
expressed on set form as 
\begin{equation}
  F_1 = 2 \frac{X_t \cap X_{t+1}}{X_t \cup X_{t+1}}
  \label{eq:f1_set}
\end{equation}
where $X_t$ denotes the set in question in timepoint $t$. In other words, the more cells are
contained between timepoints, the higher the F1 score.

As \cref{fig:tracking_quality} shows, the quality for the tracking is overall
relatively high, with only a few timepoints having bigger clusters with low F1
scores. These cells are however typically positioned at the periphery of the
SAM, at the very edges of the segmentation, and are therefore both of lesser
interest and importance for the dynamics of the stem cell niche. As described in
\cref{sec:filtering}, we have in all of our analyses excluded the tracking of
cells whose F1 scores are less than $0.30$. 

That a timepoints 40 and 48 hours in plant 2, as well as timepoint 20 hours in
plant 4 have perfect F1 scores are due to the tracking procedure failing at
these points. The tracking has therefore been manually corrected for the cells
within $30\mu$m of the highest CLV3 expressing cell.

\section{Segmentation errors}
\label{sec:data_errors_segmentation}
The data is affilited with several types of segmentation errors. For both the
nuclear and the membrane segmentation, this typically takes the form of basins
of attraction identified as either one when there in fact are multiple, or
multiple when there is in fact only one. In particular, this causes errors with
respect to the apprciation of membrane and nuclear volumes. It also indirectly
affects the tracking quality, as clumped cells in one timepoint can lead to an
erronously identified division event in the next, in case the segmentation
becomes correct. 

Segmentation errors are however the most prevalent in the peripheral regions of
the SAM, and with deeper penetration into the tissue. Generally, the membrane
segmentation holds higher quality than the nuclear one, where the latter
degrades quickly after deeper penetration than the L2. The overall quality is
also decreasing with the distance from the apex, although to a minor extent in
the L1. 

Generally, the overall quality for the membrane segmentation is high in all
three layers, although the number of errors increase with deeper tissue
penetration. For the nuclear segmentation, mostly cells from the L1 and L2 are
sufficiently reliable, with declining quality with larger distance to the CZ. 





%Discussion of quality decay with increasing d2t.


%\subsubsection{Tracking errors}
%\label{sec:data_errors_tracking}
%Manual tracking in some timepoints...
%\section{Layer Quality Assessment}



