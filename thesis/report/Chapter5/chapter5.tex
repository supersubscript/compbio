%!TEX root = ../thesis.tex
%*******************************************************************************
%****************************** Third Chapter **********************************
%*******************************************************************************
\chapter{Outlook}

% **************************** Define Graphics Path **************************
\ifpdf
\graphicspath{{Chapter5/Figs/Raster/}{Chapter5/Figs/PDF/}{Chapter5/Figs/}}
\else
\graphicspath{{Chapter5/Figs/Vector/}{Chapter5/Figs/}}
\fi

\section{Significance of results}
% Few single cells at apex maintained
Our analysis presented herein has given extensive novel insights into the dynamics of
the stem cell niche of AT. In particular, we have showed how the SAM appears to
maintain a few single cells that are truly expressing CLV3, and thus also set
the grounds for further investigations into the variance in and extent of this small
pool of cells; for example, questions we have not addressed herein are the
variability in the size of the true CLV3 domain in relationship to the overall
size and shape of the SAM itself. 

% Dynamic perspective
While previous research has primarily focused on the static view of the stem
cell niche, we have here provided parts also of the dynamic perspective --
something which has previously been severely lacking. Because of this, we are
able to better support the hypothesis of epidermal regulation in the
CZ, and have also pointed at cues of the extent of differentiation driving peripheral
signals. That is, our observed differences in functional behaviour for cells at
various distances from the CZ suggests a possible domain for where peripheral
signalling is either present, or simply begins to affect cell functionality.
This information can be of use in particular for modelling, where
understanding the extent of unknown regulatory mechanisms can aid in unraveling
their possible nature and from there guide experiments.

% 
We have also shown how the CLV3 apex does not coincide with the geometrical
apex, something which might have an impact on anisotropic growth and
phyllotaxis. This is a completely unprecedented result, which might provide
future insights into how the interplay between signalling in the CZ and PZ is orchestrated
in order for plants to undergo primordia initiation.

\section{Expansion of modelling framework}
Our models herein have taken an abstract approach, which while sufficient for
our purposes does not capture the complexity and intricacies of the real tissue.
Modern software tools are now able to generate contact maps from raw confocal
images, thus creating the means for accurately comparing model simulations to
the real tissue. This is a possible extension to the work herein, which would
also be a natural future validation step. 

Establishing models for the $>1$D case would in addition allow for validating
models relating to the combined aspect of our results. For example, is epidermal
regulation sufficient for producing both our distribution for the CLV3
intensities with respect to the distance to the apex, as well as the apparent
distributional shift between the L1 and L2? A simple model in e.g.\ 2D could
likely verify the potentiality of such an hypothesis.

Also a spatial model relating to growth rates could be established to test if
there are mechanical constraints limiting the growth rate of the apical cells in
comparison of their neighbours, and possible biochemical signals regulating
this. The combined mechanical and biochemical framework is likely also something
which will be able to provide a more complete picture of meristem growth,
relating to both scaling of the stem cell niche itself, but also how that
relates to cell features such as growth and division rates depending on their
spatial localisation.

\section{Data investigation}
% Pin1
A natural and very feasible next step in quantifying the dynamics of the genes
present in the SAM would able to also include the PIN1 reporter and observe how
this behaves in relation to the CLV3 signal. Possibly this aids in elucidating
the extent of both technical noise and what observed fluctuations are true
biological events. It also allows insight into the actual auxin transport
perspective of the setting, clearly showing that auxin transport is indeed
resumed, as well as a glimpse into how this affects the periodicity we observe
for the stem cell niche. 

% Different layers
Like for PIN1, there are multiple aspects of the data analysis not fully
investigated in this thesis. This includes more thorough research into
differences between the different layers of the SAM, and how the spatial
localisation relates to e.g.\ cell growth. In particularly the CZ of the SAM,
steady maintenance between the highest expressing cells and their contact areas
to neighbouring cells could prove to gain a better understanding of
the extent and regulation of the CLV3 expressing cells. 

The possible clustering of long-lived cells in the deeper, central regions
tending towards the OC is a fascinating suggestive discovery, which is somewhat
hampered by the difficulty to accurately identify cells within the tissue.
One possibility in this case would be to develop methods able to determine
cellular age from static images, where the plant can then be cut in the coronal
plane in order to improve imaging. This would be enabled by information about, among
other measures, relative cell wall angles between neighbouring cells. An
additional approach could be to tag cells with cell cycle identifying markers,
although also in this case it is likely to be hampered by the penetration depth
of imaging tools.

As the data in addition includes multiple segmentation errors of various types,
a reevaluation of these could help in better narrowing down the behaviour of
especially individual cell lines -- something which in our analysis is slightly
hampered due to lesser tracking quality outside of the CZ. This holds especially
true for the individual timepoints in the tracking which have been manually
corrected for, and therefore do not include cells outside of the CZ; a simple
re-tracking of these timepoints would provide both more data for the data set
overall, but would in particular enable insight into the cell line dynamics of
the subepidermis, and how they correspond to the ones in the L1. Performing an
accurate segmentation step also for plant 18 would on top of this provide a
larger data set to base the extended analysis on.

\section{New experiments}
% WUS or other tracker
Because of the difficulty in unraveling which parts of a GRN is essential for
giving it its correct functionality, it is also difficult to know which
molecules to track for in \textit{in vivo} time-lapse assays.
In the analysis of the regulation of the stem cell niche, having access to
tracker information also for the WUS would improve upon the identification of
core network dynamics significantly, although this type of data has not been
available until recently. Extending the network further, to be able to also
track also the patterning activity of auxin on top of the PIN1 data give useful
information not only of when and how much auxin transport is activated, but also
where this localisation happens in the tissue. Being able to relate this
information in particular to the behaviour of cells in particular lineages would
provide information of how the CLV3 and cell dynamics depend on auxin localisation.

% Division driving gene?
As the CLV3 level of expression does not appear to provide cues for when a cell
is to undergo a division event, identifying the driving genes causing this
event to happen is a future endeavour. In the event that no such gene exists,
and that it is predominantly mechanical signalling that determines
proliferation, the interplay between the mechanical and the chemical is perhaps
the more relevant. The question in that case would then primarily be to identify
growth-inducing agents, rather than direct division event markers.

% Wildtype
A drawback with the analysis presented herein in the reliance on meristems grown
on NPA for the analysis which is better suited for the steady state. However, as
physically removing organsm and primordia for enhancing imaging is majorly
perturbing the system in itself, doing so at this point is not feasible. The
future might hold tools for performing this type of research, which would
greatly enhance to quality and robustness by which the analysis of e.g.\ the
CLV3 distributions and behaviour over time is done.

% CLV3 peptide
Lastly, the reason for the use of a promotor tagging molecule in dsRED in our
experimental data is the size of the CLV3 peptide. As attempts to track the protein
itself leads to non-functional plant phenotypes, understanding the specific
protein patterning is as of yet practically unfeasible in a dynamic setting. The
future might hold advances in methodology that enables direct feedback of
peptide localisation, which undoubtedly would provide a major stepping stone
towards truly understanding plant development.

% What is the division driving gene?
% CLV3-WUS
% What is driving the oscillations? How does auxin regulate CLV3-WUS?
