\documentclass[10pt]{article}\usepackage[]{graphicx}\usepackage[]{color}
%% maxwidth is the original width if it is less than linewidth
%% otherwise use linewidth (to make sure the graphics do not exceed the margin)
\makeatletter
\def\maxwidth{ %
  \ifdim\Gin@nat@width>\linewidth
    \linewidth
  \else
    \Gin@nat@width
  \fi
}
\makeatother

\usepackage{Sweavel}


\usepackage{hyperref}
\usepackage{url}
\usepackage[a4paper]{geometry}
\usepackage{a4wide}
\usepackage{float}
\usepackage[english]{babel}
\usepackage[utf8]{inputenc}
\usepackage{csquotes}
\usepackage{amsmath}
\usepackage{amssymb}
\usepackage{xspace}
\usepackage[numbers]{natbib}
\bibliographystyle{unsrtnat}
\usepackage{subcaption}
\usepackage[font={small}]{caption}
\usepackage{booktabs}
\usepackage{listings}
\usepackage{cleveref}
\usepackage{lipsum}
\newcommand{\approxtext}[1]{\ensuremath{\stackrel{\text{#1}}{=}}}
\newcommand{\matr}[1]{\mathbf{#1}}
\newcommand{\partt}[2]{\ensuremath{\dfrac{\partial {#1}}{\partial {#2}}}}
\renewcommand{\d}[1]{\ensuremath{\operatorname{d}\!{#1}}} % non-italized differentials
\newcommand{\h}[0]{\ensuremath{\hbar}} % hbar
\def\changemargin#1#2{\list{}{\rightmargin#2\leftmargin#1}\item[]}
\let\endchangemargin=\endlist 
\usepackage{amsthm}
\theoremstyle{plain}
\renewcommand{\theequation}{\thesection.\arabic{equation}}
\def\changemargin#1#2{\list{}{\rightmargin#2\leftmargin#1}\item[]}
\let\endchangemargin=\endlist    
\usepackage{xcolor}
\definecolor{Red}{rgb}{0.7,0,0}
\definecolor{Blue}{rgb}{0,0,0.8}
\usepackage{verbatim}
\def\changemargin#1#2{\list{}{\rightmargin#2\leftmargin#1}\item[]}
\let\endchangemargin=\endlist
\addtolength{\oddsidemargin}{-.35in}
\addtolength{\evensidemargin}{-.35in}
\addtolength{\textwidth}{.7in}
\usepackage{multicol}

% Stephen's stuff
\newcommand{\R}{\texttt{R}}
\newcommand{\Rfunction}[1]{{\texttt{#1}}}
\newcommand{\Robject}[1]{{\texttt{#1}}}
\newcommand{\Rpackage}[1]{{\mbox{\normalfont\textsf{#1}}}}
\usepackage{xcolor}
\definecolor{Red}{rgb}{0.7,0,0}
\definecolor{Blue}{rgb}{0,0,0.8}
\hypersetup{%
pdfusetitle,
bookmarks = {true},
bookmarksnumbered = {true},
bookmarksopen = {true},
bookmarksopenlevel = 2,
unicode = {true},
breaklinks = {false},
hyperindex = {true},
colorlinks = {true},
linktocpage = {true},
plainpages = {false},
linkcolor = {Blue},
citecolor = {Blue},
urlcolor = {Red},
pdfstartview = {Fit},
pdfpagemode = {UseOutlines},
pdfview = {XYZ null null null}
}
%% Listings
\lstset{ 
language=R,                     % the language of the code
basicstyle=\footnotesize,       % the size of the fonts that are used for the code
numbers=left,                   % where to put the line-numbers
numberstyle=\tiny\color{gray},  % the style that is used for the line-numbers
stepnumber=1,                   % the step between two line-numbers. If it's 1, each line will be numbered
numbersep=5pt,                  % how far the line-numbers are from the code
backgroundcolor=\color{white},  % choose the background color. You must add \usepackage{color}
showspaces=false,               % show spaces adding particular underscores
showstringspaces=false,         % underline spaces within strings
showtabs=false,                 % show tabs within strings adding particular underscores
rulecolor=\color{black},        % if not set, the frame-color may be changed on line-breaks within not-black text (e.g. commens (green here))
tabsize=2,                      % sets default tabsize to 2 spaces
captionpos=b,                   % sets the caption-position to bottom
breaklines=true,                % sets automatic line breaking
breakatwhitespace=false,        % sets if automatic breaks should only happen at whitespace
title=\lstname,                 % show the filename of files included with \lstinputlisting;
% also try caption instead of title
keywordstyle=\color{Blue},      % keyword style
commentstyle=\color{orange},    % comment style
stringstyle=\color{Red},        % string literal style
escapeinside={\%*}{*)},         % if you want to add a comment within your code
morekeywords={*,...}            % if you want to add more keywords to the set
} 


%%% Document specific
\newcommand{\course}{Structural Biology}
\newcommand{\ass}{3}
\newcommand{\term}{Lent term 2017}
%\bibliography{pga1}

%%% Title page
\title{
  \bf \course: Assignment \ass \\[1em]
  \small{University of Cambridge}
}

\author{Henrik Åhl}
\date{\today}
\renewcommand{\textfraction}{0.05}
\renewcommand{\topfraction}{0.8}
\renewcommand{\bottomfraction}{0.8}
\renewcommand{\floatpagefraction}{0.75}

%%% Actual document
\begin{document}
\date{\today}
\maketitle
\setcounter{page}{1}


% \date{\today}
\maketitle
% \begin{abstract}
% {\bf 
%   %\begin{changemargin}{-.8cm}{-.8cm}
%   This is an abstract abstract.
% }
% \end{abstract}

\begin{multicols*}{2}
\section*{Preface}
This is an assignment report in connection to the \textit{\course}
module in the Computational Biology course at the University of Cambridge,
\term. All related code is as of \date{\today} available through a
Github repository by contacting \href{mailto:hpa22@cam.ac.uk}{hpa22@cam.ac.uk}.

\section*{Exercises}
\subsection*{1}
% this code reads in the output of get_constraints.m and compares it to the
% experimentally solved crystal structure of a protein sequence from the
% alignment. 

% Fig 3: plots min all atom distance against DI ranking (x axis).
% Fig 2: plots the distance in structure for each pair vs the rank of the pair
% Fig 3/4:  contact map overlay plotsmia
% 3: 	plot(plotRows,plotCols,'o','MarkerSize',6,'MarkerFaceColor',CRYSTAL_CONTACT_COLOR,'MarkerEdgeColor',CRYSTAL_CONTACT_COLOR);
% 4: 	plot(plotRows,plotCols,'*','MarkerSize',4,'MarkerFaceColor',EIC_COLOR,'MarkerEdgeColor',EIC_COLOR);

% Figure 1 shows the distribution of normalised frequency of the different amino acids in each respective column.
\Cref{fig:fig1} shows the frequency distribution of the weighted pairwise distances between sequences, given that their similarity fraction is $> 70 \%$.
\Cref{fig:fig2} shows the predicted protein contact map between the residues in each sequence, i.e.\ how correlated certain positions are with each other based on the evolutionarily inferred contact (EIC) scores. \Cref{fig:fig3} in turn depicts the minimal distance between the amino acids in the crystal as a function of the corresponding DI rank. Lastly, \cref{fig:fig4} shows the overlap between the experimentally observed structure and the predicted tertiary interactions.  

% Second: COntact map
% Third: Min distance
% Fourth: Contact map overlay

  \begin{figure*}[p]
    \centering
    \begin{subfigure}[b]{.49\textwidth}
      \includegraphics[width=\textwidth, trim= 4cm -2cm 4cm 2cm, clip]{../figures/fig1}
      \caption{Figure1}
      \label{fig:fig1}
    \end{subfigure}~
    \begin{subfigure}[b]{.5\textwidth}
      \includegraphics[width=\textwidth, trim= 10cm 0 10cm 0, clip]{../figures/fig2}
      \caption{Figure2}
      \label{fig:fig2}
    \end{subfigure}\\
    \begin{subfigure}[b]{.49\textwidth}
      \includegraphics[width=\textwidth, trim= 4cm -2cm 4cm 2cm, clip]{../figures/fig3}
      \caption{Figure3}
      \label{fig:fig3}
    \end{subfigure}~
      \begin{subfigure}[b]{.5\textwidth}
      \includegraphics[width=\textwidth, trim= 10cm 0cm 10cm 0, clip]{../figures/fig4}
      \caption{Figure4}
      \label{fig:fig4}
    \end{subfigure}%
    \caption{Figure of figures}
    \label{fig:init_figures}
  \end{figure*}


\bibliography{references}
\end{multicols*}
\end{document}
