% ************************** Thesis Abstract *****************************
% Use `abstract' as an option in the document class to print only the titlepage and the abstract.
\begin{abstract}
  %  Understanding plant development has implications for fields as wide-ranging as
  %  crop yield improvement and regenerative medicine. Aerial development in the
  %  plant is driven primarily by the developing center known as the shoot apical
  %  meristem (SAM), at the central zone (CZ) of which a small collection of stem
  %  cells reside and divide in order to produce cells for regeneration of tissue
  %  and new organ formation. The stem cells at the apex express the the CLAVATA-3
  %  (CLV3) gene, making the expression of which a direct identifier of the stem cell
  %  phenotype. Modern imaging tools and segmentation software have recently
  %  enabled extensive capabilities to both accurately measure cells in the SAM, as
  %  well as to follow their dynamics over time. In this thesis, we show how a
  %  small ``true'' stem cell niche is maintained in the SAM over time, and that
  %  the center of this region is not located at the geometric apex, possibly for
  %  phyllotactic priming. We also point at possible regulatory means for the SAM
  %  to maintain robustness in expression for the apical cells.


  Aerial development in a plant is driven primarily by a stem cell niche known
  as the shoot apical meristem (SAM), where stem cells at the apex express the
  CLAVATA3 (CLV3) gene. In this thesis we have analysed data on single cell
  dynamics of CLV3 expression originating from \textit{in vivo} confocal imaging
  quantified using image segmentation and cell tracking tools. We show how a
  small ``true'' population of stem cells with high CLV3 expression is maintained
  in the SAM over time. We find that the center of this expression region is not
  located exactly at the geometric apex, challenging the current textbook
  description of the system. We further show that there is a variation of CLV3
  expression on the tissue scale, while single cells have lower variability, and
  that there is a characteristic time scale in the variability, possibly
  connected to the known negative feedback regulation in the system. By
  comparing the dynamics in the data with stochastic simulations of regulatory
  models, we also point at possible regulatory means for the SAM to maintain
  robustness in expression for the apical cells. 


%  The shoot apical meristem (SAM) is a collection of stem cells that resides at
%  the tip of each shoot and provides the cells of the shoot. It is divided into
%  functional regions. The central zone (CZ) at the tip of the meristem is the
%  domain of expression of the CLAVATA3 (CLV3) gene, encoding a putative ligand
%  for a transmembrane receptor kinase, CLAVATA1, active in cells of the rib
%  meristem (RM), located just below the CZ. We show here that CLV3 restricts its
%  own domain of expression (the CZ) by preventing differentiation of peripheral
%  zone cells (PZ), which surround the CZ, into CZ cells and restricts overall
%  SAM size by a separate, long-range effect on cell division rate.  
  
  
%  Aerial development of most
%  plants are driven primarily by 
%

  %What-Why-How-Results-Significance
  % --- Background:
  % Why bother? (what problem gap are you trying to solve)
  % Is the solution already available?
  % Why now? (what would happen if we did not do this now)

  % --- Impact:
  % What will come out of your project? (expected results)
  % Who wants these results? (lead user)
  % Why do they want the results?
  % How do you plan to tell?

  % --- Evaluation form:
  % - Scientific excellence:
  % Soundness of concept and quality of objectives
  % Progress beyond the state-of-art
  % Quality and effectiveness of the science methodology and associated work plan 
  % - Potential impact through the development, dissemination and use of project results:
  % Contribution, at the European and international level, to the expected impacts
  % Appropriateness of measures for the dissemination and exploitation of project results





\end{abstract}
