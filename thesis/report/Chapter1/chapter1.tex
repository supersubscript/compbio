%!TEX root = ../thesis.tex
%*******************************************************************************
%*********************************** First Chapter *****************************
%*******************************************************************************

\chapter{Introduction}  %Title of the First Chapter

\ifpdf
    \graphicspath{{Chapter1/Figs/Raster/}{Chapter1/Figs/PDF/}{Chapter1/Figs/}}
\else
    \graphicspath{{Chapter1/Figs/Vector/}{Chapter1/Figs/}}
\fi

%********************************** %First Section  **************************************
%\section{The Shoot Apical Meristem of \textit{Arabidopsis thaliana}} %Section - 1.1 

% first letter Z is for Acronyms 
% first letter A is for Roman symbols
% first letter G is for Greek Symbols
% first letter G is for Greek Symbols
% first letter G is for Greek Symbols
% first letter X is for Other Symbols
% first letter R is for superscripts
% first letter S is for subscripts
\nomenclature[z-CZ]{CZ}{Central zone}                                          
\nomenclature[z-SAM]{SAM}{Shoot Apical Meristem}                               
\nomenclature[z-RAM]{RAM}{Root Apical Meristem}                               
\nomenclature[z-CLV3]{CLV3}{CLAVATA-3}                                         
\nomenclature[z-WUS]{WUS}{WUSCHEL}                                             
\nomenclature[z-KAN]{KAN1}{KANADI-1}                                           
\nomenclature[z-AT]{AT}{\textit{Arabidopsis thaliana}}
\nomenclature[z-L1]{L1}{Layer-1. The outermost cell layer of the SAM}
\nomenclature[z-CZ]{CZ}{Central Zone. The region harboring stem cells in the SAM.}                                                                        
\nomenclature[z-GRN]{GRN}{Gene Regulatory Network}

%********************************** %Second Section  *************************************
\section{The Shoot Apical Meristem of \textit{Arabidopsis thaliana}} %Section - 1.2
% The Shoot and the Root
Plant stem cells are governed by two developing centra -- the Shoot Apical
Meristem (SAM) and the Root Apical Meristem (RAM). The SAM is the region
responsible for development of all aerial organs in the plant, which includes
aspects of cell proliferation and specification, as well as an ability of the plant to
maintain and regulate the stem cell identity of the cells at the very apex of
the shoot. \CITE As opposed to the RAM, which has two stem cell pools in the
inside of the root, the SAM maintains a single stem cell pool centered at the
apex. It also lacks the root's cap, which protects the stem cells on the inside
of the root, whereas these in the shoot are directly exposed to the plant's surroundings. 

% What stem cells do
The stem cells at the SAM contribute to the construction of new organs and
general tissue by dividing frequently at the top and subsequently being
mechanically pushed out of the center in order to differentiate. The steady
maintenance of the stem cell niche allows for a constant production and supply
of cells that the plant utilises during both growth and repair of damaged
tissue. \CITE

% Structure composition of the plant
In a simple outline of the SAM, it can be said to consist of three core regions:
1. The \textit{central zone} (CZ), which harbors the aerial \textit{stem cell niche} 
of the plant; 2. The \textit{Rib Meristem}, which is located beneath
the CZ and consists of the cells constructing the stem of the plant; 3. The
\textit{Peripheral Zone} (PZ), where cells form organs and new tissue through
differentiation. In addition to these regions, the SAM is also often separated
into the different layers of the dermis, denoted \textit{L1} for the epidermal
layer, \textit{L2} for the sub-epidermal one, and \textit{L3} for the inner ground
and vascular tissues. For cells in both L1 and L2, proliferation happens
orthogonally to the shoot surface, i.e.\ so that cell lineages are preserved
within L1 or L2 correspondingly. In contrast, this is not the case for L3, where
cells can divide in all directions. In addition, it has been shown that the epidermis is
involved in both promoting and restricting shoot development, adding to the
notion of coordination and regulation between the different cell layers in order
to accurately direct plant growth. % Expand on this

% TODO: Lead into modelling

\section{Modelling Biological Systems} %Section - 1.2
% Systems biology
Due to the interaction of molecules in various ways, e.g. the activation or
repression of transcription by certain proteins, organismal development can be
considered in the framework of being a \textit{complex system}. In a
\textit{systems biology} setting, molecular and mechanical interactions are
treated as abstract entities, each representing some fundamental part of the
whole system in question, much like how machinery can be explained by its
separate cogs and gears working together. In a molecular setting, the typical
descriptive approach is through \textit{Gene Regulatory Networks} (GRNs), where
each component represents some molecular aspect of the system that is involved in
producing expression levels of mRNA and proteins. \CITE

% Gene regulatory networks and mathematical modelling
GRNs are commonly understood both through analytical and computational means,
where in the latter computer-generated models provides as a modern tool for
better understand the complex nature of many biological systems. Typically,
reaction kinetics are modelled using various types of \textit{Ordinary
  Differential Equations} (ODEs). However, due to the large supplies of computer
power available in the modern day, many recent studies also utilise more
computationally demanding resources such as \textit{Stochastic Differential
  Equations} (SDEs), where also the inherently random nature of molecular
motions, interactions, and processes are accounted for. The increase in
computability has also allowed for the development of spatiotemporal modelling,
where models are evaluated not only in a static context, but also in a changing
setting. A straightforward example taken directly out of the context of plant
development is how the distribution of gene expression varies during plant
growth and organ formation, both spatially and over time. Typical modelling
aspects at such a problem case involves the formulation of which genes and
molecules are imporant for the problem of interest, as well as how the
discretisation and representation of spatial elements is done. \CITE

% Model prediction and verification 
Computer models in general have two separate aims: exploration and verification.
In the former case, computer simulations can be the core for designing
experimental experiments, where observed theoretical phenomena can be
experimentally tested. An example of this is the classic example of the
\textit{Repressilator} \CITE, where researchers set up a model framework for how
oscillations could occur due to cyclic repressive interactions between three
genes. This was then verified to occur by synthetically implementing the system
in a bacterium, showing how gene expression profiles could oscillate due to the
system motif constructed. 

% TODO: 
In the latter case, which is the more prevalent in modern computational biology,
computer models are established in order to verify or support the potential of
an hypothesis due to experimental observations. \CITE



\section{Regulatory Mechanics of Plant Stem Cells} %Section - 1.2
% Patterning
In order for the plant to undergo phyllotaxis in the correct manner, and to know
when and where to initiate new organs -- primordia -- patterning of various
types play an important role. In particular genetic and molecular patterning due
to hormones, proteins and gene expression are known to be essential parts in understanding
plant development, although also patterns of stress and strain have in recent
studies been shown to play a role in determining growth. \CITE 






%While gene expression can consist of both directly functional RNA, such as tRNA, and
%mRNA, the studying gene expression in developmental biology is often done as a
%proxy for protein localisation.

% Gene expression is the process by which information from a gene is used in the
% synthesis of a functional gene product. These products are often proteins, but
% in non-protein coding genes such as transfer RNA (tRNA) or small nuclear RNA
% (snRNA) genes, the product is a functional RNA. The process of gene expression
% is used by all known life—eukaryotes (including multicellular organisms),
% prokaryotes (bacteria and archaea), and utilized by viruses—to generate the
% macromolecular machinery for life. 
% Several steps in the gene expression process may be modulated, including the
% transcription, RNA splicing, translation, and post-translational modification of
% a protein. Gene regulation gives the cell control over structure and function,
% and is the basis for cellular differentiation, morphogenesis and the versatility
% and adaptability of any organism. Gene regulation may also serve as a substrate
% for evolutionary change, since control of the timing, location, and amount of
% gene expression can have a profound effect on the functions (actions) of the
% gene in a cell or in a multicellular organism.  
% In genetics, gene expression is the most fundamental level at which the genotype
% gives rise to the phenotype, i.e. observable trait. The genetic code stored in
% DNA is "interpreted" by gene expression, and the properties of the expression
% give rise to the organism's phenotype. Such phenotypes are often expressed by
% the synthesis of proteins that control the organism's shape, or that act as
% enzymes catalysing specific metabolic pathways characterising the organism.
% Regulation of gene expression is thus critical to an organism's development.



\subsection{The role of WUSCHEL-CLAVATA interactions} %Section - 1.2.1


